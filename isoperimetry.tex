% !TeX root = ssd.tex
\section{An isoperimetric inequality on the discrete grid}
\label{sec:isoperimetry}
The isoperimetric problem on the Boolean cube $\{0,1\}^n$ proved
extremely useful in theoretical computer science. The problem is
to determine the set $S\subseteq \{0,1\}^n$ of a fixed
cardinality with the smallest ``perimeter'', or more generally,
to establish connection between the size of a set and the size
of its boundary. Here the boundary can be defined in several
ways. Considering the Boolean cube as a graph where vertices of
Hamming distance 1 are connected, the {\em edge boundary} of a
set $S$ is defined as the set of edges connecting $S$ and its
complement, while the {\em vertex boundary} consists of the
vertices outside $S$ having a neighbor in $S$.

Harper \cite{Harper1964} showed that the vertex boundary of a
Hamming ball is smallest among all sets of equal size, and the
same holds for the edge boundary of a subcube. These results can
be generalized to other cardinalities \cite{Hart1976}; see the
survey by Bezrukov \cite{Bezrukov1994}.

Consider the metric space over the set $[t]^n$ endowed 
with the Hamming distance.
Let $f$ be a concave function on the nonnegative integers 
and $1\le M<n$ be an integer. 
We consider the following
value as a generalized perimeter of a set $S\subseteq[t]^n$:
\begin{align*}
%\label{eq:nbr}
\E_{x\sim\mu}[f\left(\left|B_M(x)\cap S\right|\right)],
\end{align*}
where $B_M(x)=\{y\in[t]^n\mid\Match(x,y)\ge M\}$ is the radius
$n-M$ Hamming ball around $x$. Note that when $M=n-1$ and $f$ is
the counting function given as $f(0)=0$ and $f(l)=1$ for $l>0$
(which is concave), the above quantity is exactly the normalized
size of the vertex boundary of $S$. For other concave functions
$f$ and parameters $M$ this quantity can still be considered a
measure of how ``spread out'' the set $S$ is. We conjecture that
$n$-dimensional boxes minimize this measure in every case.

\begin{conjecture}
\label{conj:product}
Let $1\le k\le t$ and $1\le M<n$ be integers. 
Let $S$ be an arbitrary subset of $[t]^n$ of size $k^n$ and 
$P=[k]^n$. We have
\begin{align*}
\E_{x\sim\mu}[f\left(\left|B_M(x)\cap P\right|\right)]\leq
\E_{x\sim\mu}[f\left(\left|B_M(x)\cap S\right|\right)].
\end{align*}
\end{conjecture}
Even though a proof of \autoref{conj:product} remained elusive,
in \autoref{thm:isoperimetry}, we prove an approximate version
of this result, where, for technical reasons, we have to
restrict our attention to a small fraction of the coordinates.
Having this weaker result allows us to prove our communication
complexity lower bound in the next section but proving the
conjecture here would simplify this proof.

We start the technical part of this section by introducing the
notation we will use. For $x,y\in[t]^n$ and $i\in[n]$ we write
$x\sim_iy$ if $x_j=y_j$ for $j\in[n]\setminus\{i\}$. Observe
that $\sim_i$ is an equivalence relation. A set $K\subseteq
[t]^n$ is called an {\em $i$-ideal} if $x\sim_i y$, $x_i<y_i$
and $y\in K$ implies $x\in K$. We call a set $K\subseteq[t]^n$
an {\em ideal} if it is an $i$-ideal for all $i\in[n]$.

For $i\in[n]$ and $x\in[t]^n$ we define
$\down_{i}(x)=(x_1,\ldots,x_{i-1},x_i-1,x_{i+1},\ldots,x_n)$. 
We have $\down_i(x)\in[t]^n$ whenever $x_i>1$.
%
Let $K\subseteq [t]^n$ be a set, $i\in[n]$ and $2\le a\in[t]$. 
For $x\in K$, we define $\down_{i,a}(x,K)=\down_i(x)$ if 
$x_i=a$ and $\down_i(x)\notin K$ and we set 
$\down_{i,a}(x,K)=x$ otherwise. We further define
$\down_{i,a}(K)=\{\down_{i,a}(x,K)\mid x\in K\}$.
For $K\subseteq[t]^n$ and $i\in[n]$ we define
\begin{align*}
%\label{eq:down}
\down_i(K)\defeq\setbuilder{y\in[t]^n}{y_i\le\abs{\setbuilder{z\in K}{y\sim_iz}}}.
\end{align*}
Finally for $K\subseteq[t]^n$ we define
\begin{align*}
%\label{eq:down-cascade}
\down(K)\defeq\down_1\nparen{\down_2\nparen{\ldots\down_n(K)\ldots}}.
\end{align*}
The following lemma states few simple observations about these
down operations.
\begin{lemma}
\label{lem:down}
Let $K\subseteq[t]^n$ be a set and let $i,j\in[n]$ be integers. 
The following hold.
\begin{enumerate}[(i)]
\item $\down_i(K)$ can be obtained from $K$ by applying several
  operations $\down_{i,a}$.
\item $|\down_{i,a}(K)|=|K|$ for each $2\le a\le t$, 
$|\down_i(K)|=|K|$ and $|\down(K)|=|K|$.
\item $\down_i(K)$ is an $i$-ideal and
if $K$ is a $j$-ideal, then $\down_i(K)$ is also a $j$-ideal.
\item $\down(K)$ is an ideal. For any $x\in\down(K)$ we have 
$P\defeq[x_1]\times[x_2]\times\cdots\times[x_n]
\subseteq\down(K)$ and there exists a set $T\subseteq K$ with
$P= \down(T)$.
\end{enumerate}
\end{lemma}
\begin{proof}
For statement (i) notice that as long as $K$ is not 
an $i$-ideal one of the operations $\down_{i,a}$ 
will not fix $K$ and hence will decrease $\sum_{x\in K}x_i$. 
Thus a finite sequence of these operations will transform $K$ 
into an $i$-ideal. 
It is easy to see that the operations $\down_{i,a}$ preserve the
number of elements in each equivalence class of $\sim_i$, 
thus the $i$-ideal we arrive at must indeed be $\down_i(K)$.

Statement (ii) follows directly from the definitions of each of
these $\down$ operations.
 
The first claim of statement (iii), namely that $\down_i(K)$ is
an $i$-ideal, is trivial from the definition. Now assume $j\ne
i$ and $K$ is a $j$-ideal, $y\in\down_i(K)$ and $y_j>1$. To see
that $\down_i(K)$ is a $j$-ideal it is enough to prove that
$\down_j(y)\in\down_i(K)$. Since $y\in\down_i(K)$, there are
$y_i$ distinct vectors $z\in K$ that satisfy $z\sim_i y$.
Considering the vectors $\down_j(z)\sim_i\down_j(y)$ and using
that these distinct vectors are in the $j$-ideal $K$ proves that
$\down_j(y)$ is indeed contained in $\down_i(K)$.

By statement (iii), $\down(K)$ is an $i$-ideal for each $i\in
[n]$. Therefore $\down(K)$ is an ideal and the first part of
statement (iv), that is, $P\subseteq K'$ follows. We prove the
existence of suitable $T$ by induction on the dimension $n$. The
base case $n=0$ (or even $n=1$) is trivial. For the inductive
step consider $K'=\down_2(\down_3(\ldots\down_n(K)\ldots))$. As
$x\in\down(K)=\down_1(K')$, we have distinct vectors $x^{(k)}\in
K'$ for $k=1,\ldots, x_1$, satisfying $x^{(k)}\sim_1x$. Notice
that the construction of $K'$ from $K$ is performed
independently on each of the $(n-1)$-dimensional ``hyperplanes''
$S^l=\{y\in[t]^n\mid y_1=l\}$ as none of the operations
$\down_2,\ldots,\down_n$ change the first coordinate of the
vectors. We apply the inductive hypothesis to obtain the sets
$T^{(k)}\subseteq S^{x^{(k)}_1}\cap K$ such that
$\down_2(\ldots\down_n(T^{(k)})\ldots)=\{x^{(k)}_1\}
\times[x_2]\times\cdots\times[x_n]$. Using again that these sets
are in distinct hyperplanes and the operations
$\down_2,\ldots,\down_n$ act separately on the hyperplanes
$S^l$, we get for $T:=\cup_{k=1}^{x_1}T^{(k)}$ that
$$\down_2(\dots\down_n(T)\dots)=\{x^{(k)}_1\mid
k\in[x_1]\}\times[x_2]\times\cdots\times[x_n].$$ Applying
$\down_1$ on both sides finishes the proof of this last part of
the lemma.
\end{proof}

For sets $x\in[t]^n$, $I\subseteq[n]$, and integer $M\in[n]$ we
define $B_{I,M}(x)=\{y\in[t]^n\mid\Match(x_I,y_I)\ge M\}$. The
projection of $B_{I,M}$ to the coordinates in $I$ is the Hamming
ball of radius $|I|-M$ around the projection of $x$.
\begin{lemma}
\label{lem:list}
Let $I\subseteq[n]$, $M\in[n]$ and let $f$ be a concave function
on the nonnegative integers. For arbitrary $K\subseteq [t]^n$ we
have
$$\E_{x\sim\mu}[f(|B_{I,M}(x)\cap\down(K)|)]\le
\E_{x\sim\mu}[f(|B_{I,M}(x)\cap K|)].$$
\end{lemma}

\begin{proof}
By \autoref{lem:down}(i), the set $\down(K)$ can be obtained
from $K$ by a series of operations $\down_{i,a}$ with various
$i\in[n]$ and $2\le a\le t$. Therefore, it is enough to prove
that the expectation in the lemma does not increase in any one
step. Let us fix $i\in[n]$ and $2\le a\le t$. We write
$N_x=B_{I,M}(x)\cap K$ and $N'_x=B_{I,M}(x)\cap\down_{i,a}(K)$
for $x\in[t]^n$. We need to prove that
\begin{align*}
\E_{x\sim\mu}[f(|N_x|)]\ge\E_{x\sim\mu}[f(|N'_x|)].
\end{align*} 
Note that $|N_x|=|N'_x|$ whenever $i\notin I$ or
$x_i\notin\{a,a-1\}$. Thus, we can assume $i\in I$ and
concentrate on $x\in[t]^n$ with $x_i\in\{a,a-1\}$. It is enough
to prove $f(|N_x|)+f(|N_y|)\ge f(|N'_x|)+f(|N'_y|)$ for any pair
of vectors $x,y\in[t]^n$, satisfying $x_i=a$, and
$y=\down_i(x)$.

Let us fix such a pair $x,y$ and set $C=\{z\in
K\setminus\down_{i,a}(K)\emid\Match(x_I,z_I)=M\}$. Observe that
$N_x = N'_x \cup C$ and $N'_x\cap C=\emptyset$. Similarly,
observe that $N'_y = N_y \cup \down_{i,a}(C)$ and $N_y \cap
\down_{i,a}(C)=\emptyset$. Thus we have $|N'_x|=|N_x|-|C|$ and
$|N'_y|=|N_y|+|\down_{i,a}(C)|=|N_y|+|C|$.

The inequality $f(|N_x|)+f(|N_y|)\ge f(|N'_x|)+f(|N'_y|)$
follows now from the concavity of $f$, the inequalities
$|N'_x|\le|N_y|\le|N'_y|$ and the equality
$|N_x|+|N_y|=|N'_x|+|N'_y|$. Here the first inequality follows
from $\down_{i,a}(N'_x)\subseteq\down_{i,a}(N_y)$, the second
inequality and the equality comes from the observations of the
previous paragraph.
\end{proof}

\begin{lemma}
\label{lem:find-prod}
Let $K\subseteq [t]^n$ be arbitrary. 
There exists a vector $x\in K$ having at least $n/5$ 
coordinates that are greater than 
$k\defeq\frac{t}{2}\mu(K)^{5/(4n)}$.
\end{lemma}
\begin{proof}
The number of vectors that have at most $n/5$ coordinates 
greater than $k$ can be upper bounded as 
\begin{align*}
{n\choose n/5} t^{n/5} k^{4n/5} 
= t^n {n\choose n/5} (k/t)^{4n/5} 
= |K|\frac{{n \choose n/5}}{2^{4n /5}},
%\label{eq:quant}
\end{align*}
where in the last step we have substituted
$\frac{k}{t}=\frac{1}{2}\mu(K)^{5/(4n)}$ and 
$\mu(K) = |K| / t^n$.
Estimating ${n\choose n/5}\le 2^{n\entb{1/5}}$, 
we obtain that the above quantity is less than $|K|$.
Therefore, there must exists an $x\in K$ that has at least 
$n/5$ coordinates greater than $k$.
\end{proof}

\begin{theorem}
\label{thm:isoperimetry}
Let $S$ be an arbitrary subset of $[t]^n$. Let
$k=\frac{t}{2}\mu(S)^{5/(4n)}$ and $M = nk/(20t)$. There exists
a subset $T\subset S$ of size $k^{n/5}$ and $I\subset [n]$ of
size $n/5$ such that, defining $N_x=\{x'\in
T\mid\Match(x_I,x'_I)\ge M\}$, we have
\begin{enumerate}[(i)]
\item $\Pr_{x\sim\mu}[N_x=\emptyset] \le 5^{-M}$ and
\item $\E_{x\sim \mu}[\log|N_x|]\geq (n/5-M)\log k - 
n\log k /5^M$, where we take $\log 0 = -1$ to make the above 
expectation exist.
\end{enumerate}
\end{theorem}
\begin{proof}
By \autoref{lem:down}(ii), we have $|\down(S)|=|S|$. By
\autoref{lem:find-prod}, there exists an $x\in\down(S)$ having
at least $n/5$ coordinates that are greater than $k$. Let
$I\subset[n]$ be a set of $n/5$ coordinates such that $x_i\geq
k$ for a fixed $x\in\down(S)$. By \autoref{lem:down}(iv),
$\down(S)$ is an ideal and thus it contains the set
$P=\prod_iP_i$, where $P_i=[k]$ for $i\in I$ and $P_i=\{1\}$ for
$i\notin I$. Also by \autoref{lem:down}(iv), there exists a
$T\subseteq S$ such that $P = \down(T)$. We fix such a set $T$.
Clearly, $|T|=k^{n/5}$.

For a vector $x\in[t]^n$, let $h(x)$ be the number of
coordinates $i\in I$ such that $x_i\in [k]$. Note that
$\E_{x\sim \mu}[h(x)] = 4M$ and $h(x)$ has a binomial
distribution. By the Chernoff bound we have $\Pr_{x\sim
\mu}[h(x)<M] < 5^{-M}$. For $x$ with $h(x)\ge M$ we have
$|B_{I,M}(x)\cap P|\ge k^{n/5-M}$, but for $h(x)<M$ we have
$B_{I,M}(x)\cap P=\emptyset$. With the unusual convention
$\log0=-1$ we have
\begin{align*}
\E_{x\sim \mu} [\log|B_{I,M}(x)\cap P|]
&\ge\Pr[h(x)\ge M](n/5-M)\log k-\Pr[h(x)<M]\\
&>(n/5-M)\log k-n\log k/5^M
\end{align*}

We have $\down(T)=P$ and our unusual $\log$ is concave on the
nonnegative integers, so \autoref{lem:list} applies and proves
statement (ii):
\begin{align*}
\E_{x\sim \mu}[\log |N_x|] &\ge\E_{x\sim \mu} 
[\log|B_{I,M}(x)\cap P|]\\
&\ge(n/5-M)\log k - n\log k /5^M.
\end{align*}

To show statement (i), we apply \autoref{lem:list} with the
concave function $f$ defined as $f(0)=-1$ and $f(l)=0$ for all
$l>0$. We obtain that
\begin{align*}
\Pr_{x\sim\mu}[N_x=\emptyset]
&=-\E_{x\sim\mu}[f(|N_x|)]\\
&\le-\E_{x\sim\mu}[f(|B_{I,M}(x)\cap P|)]\\
&=\Pr_{x\sim\mu}[B_{I,M}(x)\cap P=\emptyset]\\
&<5^{-M}.
\end{align*}
This completes the proof.
\end{proof}
