% !TeX root = ssd.tex
\section{Lower bound for single round protocols}
\label{sec:elementary}
In this section we give an combinatorial proof that any single
round randomized protocol for the exists-equal problem with
parameters $n$ and $t=4n$ has complexity $\Omega(n\log n)$ if
its error probability is at most $1/3$. As pointed out in the
Introduction, to our knowledge this is the fist established case
when solving the OR of $n$ instances of a communication problem
requires strictly more than $n$ times the complexity needed to
solve a single such instance.

We start with with a simple and standard reduction from the
randomized protocol to the deterministic one and further to a
large set of inputs that makes the first (and in this case only)
message fixed. These steps are also used in the general round
elimination argument therefore we state them in general form.

Let $\epsilon>0$ be a small constant and let $P$ be an
$1/3$-error randomized protocol for the exists-equal problem
with parameters $n$ and $t=4n$. We repeat the protocol $P$ in
parallel taking the majority output, so that the number of
rounds does not change, the length of the messages is multiplied
by a constant and the error probability decreases below
$\epsilon$. Now we fix the coins of of this $\epsilon$-error
protocol in a way to make the resulting deterministic protocol
err on at most $\epsilon$ fraction of the possible inputs.
Denote the deterministic protocol we obtain by $Q$.

\begin{lemma}
\label{lem:determine-s}
Let $Q$ be a deterministic protocol for the $\EE_n$ problem that
makes at most $\epsilon$ error on the uniform distribution.
Assume Alice sends the first message of length $c$. There exists
an $S\subset [t]^n$ of size $\mu(S)=2^{-c-1}$ such that the
first message of Alice is fixed when $x\in S$ and we have
$\Pr_{y\sim \mu}[Q(x,y)\neq\EE(x,y)]\leq 2\epsilon$ for all
$x\in S$.
\end{lemma}

\begin{proof}
Note that the quantity $e(x)=\Pr_{y\sim
\mu}[Q(x,y)\neq\EE(x,y)]$, averaged over all $x$, is the error
probability of $Q$ on the uniform input, hence is at most
$\epsilon$. Therefore for at least half of $x$, we have
$e(x)\leq 2\epsilon$. The first message of Alice partitions this
half into at most $2^c$ subsets. We pick $S$ to consist of
$t^n/2^{c+1}$ vectors of the same part: at least one part must
have this many elements.
\end{proof}

We fix a set $S$ as guaranteed by the lemma. We assume we
started with a single round protocol, so $Q(x,y)=Q(x',y)$
whenever $x,x'\in S$. Indeed, Alice sends the same message by
the choice of $S$ and then the output is determined by Bob, who
has the same input in the two cases.

We call a pair $(x,y)$ {\em bad} if $x\in S$, $y\in[t]^n$ and
$Q$ errs on this input, i.e., $Q(x,y)\ne\EE(x,y)$. Let $b$ be
the number of bad pairs. By \autoref{lem:determine-s} each
$x\in|S|$ is involved in at most $2\epsilon t^n$ bad pairs, so
we have $$b\le2\epsilon|S|t^n.$$
%
We call a triple $(x,x',y)$ {\em bad} if $x,x'\in S$,
$y\in[t]^n$, $\EE(x,y)=1$ and $\EE(x',y)=0$. The proof is based
on double counting the number $z$ of bad triples.
%
Note that for a bad triple $(x,x',y)$ we have $Q(x,y)=Q(x',y)$
but $\EE(x,y)\ne\EE(x',y)$, so $Q$ must err on either $(x,y)$ or
$(x',y)$ making one of these pairs bad. Any pair (bad or not) is
involved in at most $|S|$ bad triples, so we have
\begin{align*}
z\le b|S|\le2\epsilon|S|^2t^n.
\end{align*}

Let us fix arbitrary $x,x'\in S$ with $\Match(x,x')\le n/2$. We
estimate the number of $y\in[t]^n$ that makes $(x,x',y)$ a bad
triple. Such a $y$ must have $\Match(x,y)>\Match(x',y)=0$. To
simplify the calculation we only count the vectors $y$ with
$\Match(x,y)=1$. The match between $y$ and $x$ can occur at any
position $i$ with $x_i\ne x'_i$. After fixing the coordinate
$y_i=x_i$ we can pick the remaining coordinates $y_j$ of $y$
freely as long as we avoid $x_j$ and $x'_j$. Thus we have
\begin{align*}
|\{y\emid(x,x'y)\hbox{ is bad}\}|\ge
(n-\Match(x,y))(t-2)^{n-1}\ge(n/2)(t-2)^{n-1}>t^n/14,
\end{align*}
where in the last inequality we used $t=4n$. Let $s$ be the size
of the Hamming ball
$B_{n/2}(x)=\{y\in[t]^n\emid\Match(x,y)>n/2\}$. By the Chernoff
bound we have $s<t^n/n^{n/2}$ (using $t=4n$ again). For a fixed
$x$ we have at least $|S|-s$ choices for $x'\in S$ with
$\Match(x,x')\le n/2$ when the above bound for triples apply.
Thus we have $$z\ge|S|(|S|-s)t^n/14.$$ Combining this with the
lower bound on the number of bad triples we get
$$28\epsilon|S|\ge|S|-s.$$

Therefore we conclude that we either have large error
$\epsilon>1/56$ or else we have $|S|\le2s<2t^n/n^{n/2}$. As we
have $|S|=t^n/2^{c+1}$ the latter possibility implies 
$$c\ge n\log n/2-2.$$ Summarizing we have the following.

\begin{theorem} \label{thm:singleround} A single round
randomized protocol for $\EE_n$ with error probability $1/3$
has complexity $\Omega(n\log n)$.
A single round deterministic protocol for $\EE_n$ that errs on
at most $1/56$ fraction of the inputs has complexity at least
$n\log n/2-2$.
\end{theorem}