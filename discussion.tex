% !TeX root = ssd.tex
\section{Discussion}
\label{sec:discussion}
The $r$-round protocol we gave in \autoref{sec:upperbound}
solves the sparse set disjointness problem in $O(k\log^{(r)}k)$
total communication. As we proved in \autoref{sec:lowerbound}
this is optimal. The same, however, cannot be said of the error
probability. With the same protocol, but with more careful
setting of the parameters the exponentially small error
$O(2^{-\sqrt k})$ of the $\log^*k$-round protocol can be further
decreased to $2^{-k^{1-o(1)}}$.

For small (say, constant) values of $r$ this protocol cannot
achieve exponentially small error error without the increase in
the complexity if the universe size $m$ is unbounded. But if $m$
is polynomial in $k$ (or even slightly larger,
$m=\exp^{(r)}(O(\log^{(r)}k))$), we can replace the last round
of the protocol by one player deterministically sending his or
her entire ``current set'' $S_r$. With careful setting of the
parameters in other rounds, this modified protocol has the same
$O(k\log^{(r)}k)$ complexity but the error is now exponentially
small: $O(2^{-k/\log k})$. Note that in our lower bound on the
$r$-round complexity of the sparse set disjointness we we use
the exists-equal problem with parameters $n=k$ and $t=4k$. This
corresponds to the universe size $m=tn=4k^2$. In this case any
protocol solving the exists-equal problem with $1/3$ error can
be strengthened to exponentially small error using the same
number of rounds and only a constant factor more communication.

Our lower and upper bounds match for the exists-equal problem
with parameters $n$ and $t=\Omega(n)$, since the upper bounds
were established without any regard of the universe size, while
the lower bounds worked for $t=4n$. Extensions of the techniques
presented in this paper give matching bounds also in the case
$3\le t<n$, where the $r$-round complexity is
$\Theta(n\log^{(r)}t)$ for $r\le\log^*t$. Note, however, that in
this case one needs to consider significantly more complicated
input distributions and a more refined isoperimetric inequality,
that does not permit arbitrary mismatches. The $\Omega(n)$ lower
bound applies for the exists-equal problem of parameters $n$ and
$t\ge3$ regardless of the number of rounds, as the disjointness
problem on a universe of size $n$ is a sub-problem. For $t=2$
the situation is drastically different, the exists-equal problem
with $t=2$ is equivalent to a single equality problem.

Finally a remark on using the joint random source model of
randomized protocols throughout the paper. By a result of Newman
\cite{Newman1991} our protocols of \autoref{sec:upperbound} can be
made to work in private coin model (or even if one of the
players is forced to behave deterministically) by increasing the
first message length by $O(\log\log(N)+\log(1/\epsilon))$ bits,
where $N= {m \choose k}$ is the number of possible inputs. In
our case this means adding the term $O(\log\log m)+o(k)$ to our
bound of $\smash{O(k\log^{(r)}k)}$, since our protocols make at least
$\exp(-k/\log k)$ error. This additional cost is insignificant
for reasonably small values of $m$, but it is necessary for
large values as the equality problem, which is an instance of
disjointness, requires $\Omega(\log \log m)$-bits in the private
coin model.

Note also that we achieve a super-linear increase in the
communication for OR of $n$ instances of equality even in the
private coin model for $r=1$. For $r\geq 2$, no such increase
happens in the private coin model as communication complexity of
$\EE^t_n$ is at most $O(n\log\log t)$ however a single equality
problem requires $\Omega(\log \log t)$ bits.
