% !TeX root = ssd.tex
\section{Introduction}
In a two player communication problem the players, named Alice
and Bob, receive separate inputs, $x$ and $y$, and they
communicate in order to compute the value $f(x,y)$ of a function
$f$ known to both of them. 
In an $r$-round protocol, the players can take at most $r$
turns alternately sending each other a message 
(that is, a bit string) and the last
player to receive a message declares the output of the protocol.
A protocol can be {\em deterministic} or {\em randomized}; in
the latter case the players can base their actions on a common
random source and we measure the {\em error probability}: the
maximum over inputs $(x,y)$, of the probability that the output
of the protocol differs from $f(x,y)$.

\subsection{Sparse set disjointness}
Set disjointness is perhaps the most studied problem in
communication complexity. In the most standard version Alice and
Bob receive a subset of $[m]\defeq\{1,\ldots,m\}$ each, with the
goal of deciding whether their sets intersect or not. The
primary question is whether the players can improve on the
trivial deterministic protocol, where the first player sends the
entire input to the other player, thereby communicating $m$
bits. The first lower bound on the randomized complexity of this
problem was given in \cite{BabaiFS1986} by Babai, Frankl and Simon, 
who showed that any $\eps$-error protocol for disjointness must
communicate $\Omega(\sqrt{m})$ bits. The tight bound of
$\Omega(m)$-bits was first given by Kalyanasundaram and
Schnitger \cite{KalyanasundaramS1992} and was later simplified by
Razborov \cite{Razborov1992} and 
Bar-Yossef, Jayram, Kumar and Sivakumar \cite{Bar-YossefJKS2004}.

In the sparse set disjointness problem $\DISJ_k^m$, the sets
given to the players are guaranteed to have at most $k$
elements. The deterministic communication complexity of this
problem is well understood. The trivial protocol, where Alice
sends her entire input to Bob solves the problem in one round
using $O(k\log(m/k) + k)$ bits. On the other hand, an
$\Omega(k\log(m/k) + k)$ bit total communication lower bound can be
shown even for protocols with an arbitrary number of rounds, say
using the rank method; see \cite{KushilevitzN1997}, page 175.

The randomized complexity of the problem is far more subtle. The
result of Kalyanasundaram and Schnitger cited above immediately 
imply an $\Omega(k)$ lower bound for this version of the problem. 
The folklore $1$-round protocol
solves the problem using $O(k\log k)$ bits, wherein Alice sends
$O(\log k)$-bit hashes for each element of her set. Håstad and
Wigderson \cite{HastadW2007} gave a protocol that matches the
$\Omega(k)$ lower bound mentioned above. Their $O(k)$-bit
randomized protocol runs in $O(\log k)$-rounds and errs with a
small constant probability. In \autoref{sec:upperbound}, we
improve this protocol to run in $\log^*k$ rounds, still with
$O(k)$ total communication, but with exponentially small error
in $k$. We also present an $r$-round protocol for any
$r<\log^*k$ with total communication $O(k\log^{(r)}k)$ and error
probability well below $1/k$; see \autoref{thm:ub}. (Here
$\log^{(r)}$ denotes the iterated logarithm function, see
\autoref{sec:preliminaries}.) As the exists-equal problem with parameters
$t$ and $n$ (see below) is a special case of $\DISJ_n^{tn}$, our
lower bounds for the exists-equal problem (see below) show that
complexity of this algorithm is optimal for any number
$r\le\log^*k$ of rounds, even if we allow the much larger error
probability of $1/3$. Buhrman et al.~\cite{BuhrmanGMW2012} and
Woodruff \cite{Woodruff2008} (as presented in \cite{Patrascu2009})
show an $\Omega(k\log k)$ lower bound for $1$-round complexity
of $\DISJ^m_k$ by a reduction from the indexing problem (this 
reduction was also mentioned in \cite{DasguptaKS12}). We
note that these lower bounds do not apply to the exists-equal
problem, as the input distribution they use generates instances
inherently specific to the disjointness problem; furthermore
this distribution admits a $O(\log k)$ protocol in two rounds.

\subsection{The exists-equal problem}
In the equality problem Alice and Bob receive elements $x$ and
$y$ of a universe $[t]$ and they have to decide whether $x=y$.
We define the two player communication game exists-equal with
parameters $t$ and $n$ as follows. Each player is given an
$n$-dimensional vector from $[t]^n$, namely $x$ and $y$. The
value of the game is one if there exists a coordinate $i\in[n]$
such that $x_i = y_i$, zero otherwise. Clearly, this problem is
the OR of $n$ independent instances of the equality problem.

The direct sum problem in communication complexity is the study
of whether $n$ instances of a problem can be solved using less
than $n$ times the communication required for a single instance
of the problem. This question has been studied extensively for
specific communication problems as well as some class of
problems \cite{ChakrabartiSWY2001, JainRS2003, JainRS2005,
Ben-AroyaRW2008, Gavinsky2008, JainKN2008, HarshaJMR2010, BarakBCR2010}.
The so called direct sum approach is a very powerful tool to
show lower bounds for communication games. In this approach, one
expresses the problem at hand, say as the OR of $n$ instances of
a simpler function and the lower bound is obtained by combining
a lower bound for the simpler problem with a direct sum
argument. For instance, the two-player and multi-player
disjointness bounds of \cite{Bar-YossefJKS2004}, the lopsided set
disjointness bounds \cite{Patrascu2011}, and the lower bounds for
several communication problems that arise from streaming
algorithms \cite{JayramW2009, MagniezMN2010} are a few examples of
results that follow this approach.

Exists-equal with parameters $t$ and $n$ is a special case of
$\DISJ_n^{tn}$, so our protocols in \autoref{sec:upperbound}
solve exists-equal. We show that when $t=\Omega(n)$ these
protocols are optimal, namely every $r$-round randomized
protocol ($r\le\log^*n$) with at most $1/3$ error error
probability needs to send at least one message of size
$\Omega(n\log^{(r)}n)$ bits. See \autoref{thm:main}. Our result
shows that computing the OR of $n$ instances of the equality
problem requires {\em strictly more} than $n$ times the
communication required to solve a single instance of the
equality problem when the number of rounds is smaller than
$\log^* n-O(1)$. Recall that the equality problem admits an
$\epsilon$-error $\log(1/\epsilon)$-bit one-round protocol in
the common random source model.

For $r=1$, our result implies that to compute the OR of $n$
instances of the equality problem with {\em constant
probability}, no protocol can do better than solving each
instance of the equality problem with {\em high probability} so
that the union bound can be applied when taking the OR of the
computed results. The single round case of our lower bound also
generalizes the $\Omega(n\log n)$ lower bound of Molinaro et
al.\ \cite{MolinaroWY2013} for the one round communication
problem, where the players have to find all the answers of $n$
equality problems, outputting an $n$ bit string.

\subsection{Lower bound techniques}

We obtain our general lower bound via a round elimination
argument. In such an argument one assumes the existence of a
protocol $P$ that solves a communication problem, say $f$, in
$r$ rounds. By suitably modifying the internals of $P$, one
obtains another protocol $P'$ with $r-1$ rounds, which typically
solves smaller instances of $f$ or has larger error than $P$.
Iterating this process, one obtains a protocol with zero rounds.
If the protocol we obtain solves non-trivial instances of $f$
with good probability, we conclude that we have arrived at a
contradiction, therefore the protocol we started with, $P$,
cannot exist. Although round elimination arguments have been
used for a long time, our round elimination lemma is the first
to prove a {\em super-linear} communication lower bound in the
number of primitive problems involved, obtaining which requires
new and interesting ideas.

The general round elimination presented in
\autoref{sec:lowerbound} is very involved, but the lower bound
on the one-round protocols can also be obtained in a more
elementary way. As the one round case exhibits the most dramatic
super-linear increase in the communication cost and also
generalizes the lower bound in \cite{MolinaroWY2013}, we include
this combinatorial argument separately in
\autoref{sec:elementary}, see \autoref{thm:singleround}.

At the heart of the general round elimination lemma is a new
isoperimetric inequality on the discrete cube $[t]^n$ endowed
with the Hamming distance. We present this result,
\autoref{thm:isoperimetry}, in \autoref{sec:isoperimetry}. To
the best of our knowledge, the first isoperimetric inequality on
this metric space was proven by Lindsey in \cite{Lindsey1964},
where the subsets of $[t]^n$ of a certain size with the so
called minimum induced-edge number were characterized. This
result was rediscovered in \cite{KleitmanKR1971} and
\cite{Clements1971} as well. See \cite{AzizogluO2003} for a
generalization of this inequality to universes which are
$n$-dimensional boxes with arbitrary side lengths. In
\cite{BollobasL1991}, Bollobás et al.\ study isoperimetric
inequalities on $[t]^n$ endowed with the $\ell_1$ distance. For
the purposes of our proof we need to find sets $S$ that minimize
a substantially more complicated measure. This measure also
captures how spread out $S$ is and can be described roughly as
the average over points $x\in[t]^n$ of the logarithm of the
number of points in the intersection of $S$ and a Hamming ball
around $x$.

\subsection{Related work} In \cite{MiltersenNSW1998}, a round
elimination lemma was given, which applies to a class of
problems with certain self-reducibility properties. The lemma is
then is used to get lower bounds for various problems including
the greater-than and the predecessor problems. This result was
later tightened in \cite{Sen2003} to get better bounds for the
aforementioned problems. Different round elimination arguments
were also used in \cite{KarchmerW1990, HalstenbergR1988,
NisanW1993,Miltersen1994, DurisGS1987,BeameF2001} for various
communication complexity lower bounds and most recently in
\cite{BrodyC2009} and \cite{BrodyCRVW2010} for obtaining lower
bounds for the gapped Hamming distance problem.

In parallel and independent of the present form of this paper
Brody et al.\ \cite{BrodyCK2012} have also established an
$\Omega(n\log^{(r)}n)$ lower bound for the $r$-round
communication complexity of the exists-equal problem with
parameter $n$. Their result applies for protocols with a
polynomially small error probability like $1/n$. This stronger
assumption on the protocol allows for simpler proof techniques,
namely the information complexity based direct sum technique
developed in several papers including
\cite{Ablayev1996,ChakrabartiSWY2001}, but it is not enough to
create an example where solving the OR of $n$ communication
problems requires more than $n$ times the communication of
solving a single instance. Indeed, even in the shared random
source model one needs $\log n$ bits of communication
(independent of the number of rounds) to achieve $1/n$ error in
a single equality problem.

\subsection{Structure of the paper}

We start in \autoref{sec:upperbound} with our protocols for the
sparse set disjointness. Note that the exists-equal problem is a
special case of sparse set disjointness, so our protocols work
also for the exists-equal problem. In the rest of the paper we
establish matching lower bounds showing that the complexity of
our protocols are within a constant factor to optimal for both
the exists-equal and the sparse set disjointness problems, and
for any number of rounds. In \autoref{sec:elementary} we give an
elementary proof for the case of single round protocols. In
\autoref{sec:isoperimetry} we develop our isoperimetric
inequality and in \autoref{sec:lowerbound} we use it in our
round elimination proof to get the lower bound for multiple
round protocols. Finally in \autoref{sec:discussion} we point
toward possible extensions of our results.
