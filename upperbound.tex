% !TeX root = ssd.tex
\section{The upper bound}
\label{sec:upperbound}
Recall that in the communication problem $\DISJ_k^m$, each of
the two players is given a subset of $[m]$ of size at most $k$
and they communicate in order to determine whether their sets
are disjoint or not. In 1997, H\aa stad and Wigderson
\cite{ParnafesIRWA1997,HastadW2007} gave a probabilistic protocol
that solves this problem with $O(k)$ bits of communication and
has constant one-sided error probability. The protocol takes
$O(\log k)$ rounds. Let us briefly review this protocol as this
is the starting point of our protocol.

Let $S,T\subseteq[m]$ be the inputs of Alice and Bob. Observe
that if they find a set $Z$ satisfying $S\subseteq Z\subseteq
[m]$, then Bob can replace his input $T$ with $T'=T\cap Z$ as
$T'\cap S=T\cap S$. The main observation is that if $S$ and $T$
are disjoint, then a random set $Z\supseteq S$ will intersect
$T$ in a uniform random subset, so one can expect
$|T'|\approx|T|/2$. In the H\aa stad-Wigderson protocol the
players alternate in finding a random set that contains the
current input of one of them, effectively halving the other
player's input. If in this process the input of one of the
players becomes empty, they know the original inputs were
disjoint. If, however, the sizes of their inputs do not show the
expected exponential decrease in time, then they declare that
their inputs intersect. This introduces a small one sided error.
Note that one of the two outcomes happens in $O(\log k)$ rounds.
An important observation is that Alice can describe a random set
$Z\supseteq S$ to Bob using an expected $O(|S|)$ bits by making
use of the joint random source. This makes the total
communication $O(k)$.

In our protocol proving the next theorem, we do almost the same,
but we choose the random sets $Z\supseteq S$ not uniformly, but
from a biased distribution favoring ever smaller sets. This
makes the size of the input sets of the players decrease much
more rapidly, but describing the random set $Z$ to the other
player becomes more costly. By carefully balancing the
parameters we optimize for the total communication given any
number of rounds. When the number of rounds reaches
$\log^*k-O(1)$ the communication reaches its minimum of $O(k)$
and the error becomes exponentially small.

\begin{theorem}\label{thm:ub}
For any $r\leq \log^*k$, there is an $r$-round probabilistic
protocol for $\DISJ^m_k$ with $O(k\log^{(r)}k)$ bits total
communication. There is no error for intersecting input sets,
and the probability of error for disjoint sets can be made
$O(1/\exp^{(r)}(c\log^{(r)} k)+ \exp(-\sqrt k))\ll 1/k$ for any
constant $c > 1$.

For $r=\log^*k-O(1)$ rounds this means an $O(k)$-bit protocol
with error probability $O(\exp(-\sqrt k))$.
\end{theorem}

\begin{proof}
We start with the description of the protocol. Let $S_0$ and
$S_1$ be the input sets of Alice and Bob, respectively. For
$1\le i\le r$, $i$ even Alice sends a message describing a set
$Z_i\supset S_i$ based on her ``current input'' $S_i$ and Bob
updates his ``current input'' $S_{i-1}$ to $S_{i+1}\defeq
S_{i-1}\cap Z_i$. In odd numbered rounds the same happens with
the role of Alice and Bob reversed. We depart from the H\aa
stad-Wigderson protocol in the way we choose the sets $Z_i$:
Using the shared random source the players generate $l_i$ random
subsets of $[m]$ containing each element of $[m]$ independently
and with probability $p_i$. We will set these parameters later.
The set $Z_i$ is chosen to be the first such set containing
$S_i$. Alice or Bob (depending on the parity of $i$) sends the
index of this set or ends the protocol by sending a special
error signal if none of the generated sets contain $S_i$. The
protocol ends with declaring the inputs disjoint if the error
signal is never sent and we have $S_{r+1}=\emptyset$. In all
other cases the protocol ends with declaring ``not disjoint''.

This finishes the description of the protocol except for the
setting of the parameters. Note that the error of the protocol
is one-sided: $S_0\cap S_1=S_i\cap S_{i+1}$ for $i\le r$, so
intersecting inputs cannot yield $S_{r+1}=\emptyset$.

We set the parameters (including $k_i$ used in the analysis) 
as follows:
\begin{align*}
u&=(c+1)\log^{(r)}k,\\
p_i&=\frac1{\exp^{(i)}u}&\hbox{for }1\le i\le r,\\
l_1&=k\exp(ku),\\
l_i&=k2^{k/2^{i-4}}&\hbox{for }2\le i\le r,\\
k_0&=k_1=k,\\
k_i&=\frac k{2^{i-4}\exp^{(i-1)}u}&\hbox{for }2\le i\le r,\\
k_{r+1}&=0.
\end{align*}

The message sent in round $i>1$ has length
$\lceil\log(l_i+1)\rceil<k/2^{i-4}+\log k+1$, thus the total
communication in all rounds but the first is $O(k)$. The length
of the first message is $\lceil\log(l_1+1)\rceil\le ku+\log
k+1$. The total communication is $O(ku)=O(ck\log^{(r)}k)$ as
claimed (recall that $c$ is a constant).

Let us assume the input pair is disjoint. To estimate the error
probability we call round $i$ {\em bad} if an error message is
sent or a set $S_{i+1}$ is created with $|S_{i+1}|>k_{i+1}$. If
no bad round exists we have $S_{r+1}=\emptyset$ and the protocol
makes no error. In what follows we bound the probability that
round $i$ is bad assuming the previous rounds are not bad and
therefore having $|S_j|\le k_j$ for $0\le j\le i$.

The probability that a random set constructed in round $i$
contains $S_i$ is $p_i^{-|S_i|}\ge p_i^{-k_i}$. The probability
that none of the $l_i$ sets contains $S_i$ and thus an error
message is sent is therefore at most
$(1-p_i^{k_i})^{l_i}<e^{-k}$.

If no error occurs in the first bad round $i$, then
$|S_{i+1}|>k_{i+1}$. Note that in this case $S_{i+1}=S_{i-1}\cap
Z_i$ contains each element of $S_{i-1}$ independently and with
probability $p_i$. This is because the choice of $Z_i$ was based
on it containing $S_i$, so it was independent of its
intersection with $S_{i-1}$ (recall that $S_i\cap
S_{i-1}=S_1\cap S_0=\emptyset$). For $i<r$ we use the Chernoff
bound. The expected size of $S_{i+1}$ is $|S_{i-1}|p_i\le
k_{i-1}p_i\le k_{i+1}/2$, thus the probability of
$|S_{i+1}|>k_{i+1}$ is at most $2^{-k_{i+1}/4}$. Finally for the
last round $i=r$ we use the simpler estimate $p_rk_{r-1}\le
k/\exp^{(r)}u$ for $|S_{r+1}|>k_{r+1}=0$.

Summing over all these estimates we obtain the following error
bound for our protocol:
$$\Pr[\hbox{error}]\le re^{-k}+\frac k{\exp^{(r)}u}+
\sum_{i=2}^r2^{-k_i/4}.$$
In case $k_r\ge4\sqrt k$ this error estimate proves the theorem.
In case $k_r<4\sqrt k$ we need to make a minor adjustments in
the setting of our parameters. We take $j$ to be the smallest
value with $k_j<4\sqrt k$, modify the parameters for round $j$
and stop the protocol after this round declaring ``disjoint'' if
$S_{j+1}=\emptyset$ and ``intersecting'' otherwise. The new
parameters for round $j$ are $k'_j=4\sqrt k$, $p'_j=2^{-2\sqrt
k}$, $l'_j=k2^{8k}$. This new setting of the parameters makes
the message in the last round linear in $k$, while both the
probability that round $j-1$ is bad because it makes
$|S_j|>k'_j$, or the probability that round $j$ is bad for any
reason (error message or $S_{j+1}\ne\emptyset$) is $O(2^{-\sqrt
k})$. This finishes the analysis of our protocol.
\end{proof}
